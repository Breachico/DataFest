\documentclass[]{article}
\usepackage{lmodern}
\usepackage{amssymb,amsmath}
\usepackage{ifxetex,ifluatex}
\usepackage{fixltx2e} % provides \textsubscript
\ifnum 0\ifxetex 1\fi\ifluatex 1\fi=0 % if pdftex
  \usepackage[T1]{fontenc}
  \usepackage[utf8]{inputenc}
\else % if luatex or xelatex
  \ifxetex
    \usepackage{mathspec}
  \else
    \usepackage{fontspec}
  \fi
  \defaultfontfeatures{Ligatures=TeX,Scale=MatchLowercase}
\fi
% use upquote if available, for straight quotes in verbatim environments
\IfFileExists{upquote.sty}{\usepackage{upquote}}{}
% use microtype if available
\IfFileExists{microtype.sty}{%
\usepackage{microtype}
\UseMicrotypeSet[protrusion]{basicmath} % disable protrusion for tt fonts
}{}
\usepackage[margin=1in]{geometry}
\usepackage{hyperref}
\hypersetup{unicode=true,
            pdftitle={Participant Info \& Guidelines},
            pdfborder={0 0 0},
            breaklinks=true}
\urlstyle{same}  % don't use monospace font for urls
\usepackage{graphicx,grffile}
\makeatletter
\def\maxwidth{\ifdim\Gin@nat@width>\linewidth\linewidth\else\Gin@nat@width\fi}
\def\maxheight{\ifdim\Gin@nat@height>\textheight\textheight\else\Gin@nat@height\fi}
\makeatother
% Scale images if necessary, so that they will not overflow the page
% margins by default, and it is still possible to overwrite the defaults
% using explicit options in \includegraphics[width, height, ...]{}
\setkeys{Gin}{width=\maxwidth,height=\maxheight,keepaspectratio}
\IfFileExists{parskip.sty}{%
\usepackage{parskip}
}{% else
\setlength{\parindent}{0pt}
\setlength{\parskip}{6pt plus 2pt minus 1pt}
}
\setlength{\emergencystretch}{3em}  % prevent overfull lines
\providecommand{\tightlist}{%
  \setlength{\itemsep}{0pt}\setlength{\parskip}{0pt}}
\setcounter{secnumdepth}{0}
% Redefines (sub)paragraphs to behave more like sections
\ifx\paragraph\undefined\else
\let\oldparagraph\paragraph
\renewcommand{\paragraph}[1]{\oldparagraph{#1}\mbox{}}
\fi
\ifx\subparagraph\undefined\else
\let\oldsubparagraph\subparagraph
\renewcommand{\subparagraph}[1]{\oldsubparagraph{#1}\mbox{}}
\fi

%%% Use protect on footnotes to avoid problems with footnotes in titles
\let\rmarkdownfootnote\footnote%
\def\footnote{\protect\rmarkdownfootnote}

%%% Change title format to be more compact
\usepackage{titling}

% Create subtitle command for use in maketitle
\newcommand{\subtitle}[1]{
  \posttitle{
    \begin{center}\large#1\end{center}
    }
}

\setlength{\droptitle}{-2em}

  \title{Participant Info \& Guidelines}
    \pretitle{\vspace{\droptitle}\centering\huge}
  \posttitle{\par}
  \subtitle{\url{https://chicodatafest.netlify.com/}}
  \author{}
    \preauthor{}\postauthor{}
    \date{}
    \predate{}\postdate{}
  

\begin{document}
\maketitle

\textbf{Note: This document may be updated as the event approaches; any
major updates will be clearly marked.}

This is a long document, please take a moment to read over it carefully.

\hypertarget{location}{%
\section{Location}\label{location}}

DataFest 2019 @ CSU, Chico will start with registration and end with
awards ceremony in
\href{https://www.csuchico.edu/aaspace/commonspaces/selv/index.shtml}{Selvester's
Cafe}.

\hypertarget{directions-maps}{%
\subsubsection{Directions \& Maps}\label{directions-maps}}

\begin{itemize}
\tightlist
\item
  Selvester's Cafe

  \begin{itemize}
  \tightlist
  \item
    Is in the center of campus, along the creek
  \item
    For some reason it's labelled Selvester's Cafe-By-The-Creek
  \item
    On a map:
    \url{https://www.csuchico.edu/maps/campus/?id=1193\#!ct/28514?mc/39.72979688502985,-121.84459609910849?z/18?lvl/0}
  \end{itemize}
\end{itemize}

\hypertarget{transportation-parking}{%
\subsubsection{Transportation \& Parking}\label{transportation-parking}}

\begin{itemize}
\tightlist
\item
  There is metered parking (free on weekends) on Citrus Ave and Legion
  Ave.
\item
  There is additional free street parking near the Gateway Science
  Museum.
\item
  Please don't park in the Bidwell Mansion parking lot or in the
  residential areas.
\item
  CSU, Chico offers some
  \href{https://www.csuchico.edu/parking/wheretopark.shtml}{further
  advice about parking}.
\item
  We ask that you carpool with teammates as parking on campus can be
  tricky.
\item
  Here is a map of
  \href{https://www.csuchico.edu/maps/campus/?id=1193\#!ce/28506,32116,36245?ct/28506,36245,36247,36246,36473,36472,32115,32092?mc/39.72895946407476,-121.8430553376675?z/15?lvl/0}{some
  parking spots} on campus.
\end{itemize}

\hypertarget{who-is-at-asa-datafest-chico-state-2019}{%
\subsection{Who is at ASA DataFest @ Chico State
2019?}\label{who-is-at-asa-datafest-chico-state-2019}}

\begin{itemize}
\tightlist
\item
  46 participants in 11 teams!
\item
  8 consultants
\item
  4 judges
\end{itemize}

You should receive an email from the DataFest organizers with your team
information, this document as a PDF, and information on how to join our
Slack workspace. Contact us if you have not received this information.

\hypertarget{schedule}{%
\subsection{Schedule}\label{schedule}}

The schedule is at \url{https://chicodatafest.netlify.com/schedule/}.

Registration opens at 5pm on Friday at Selvester's Cafe.

You are of course free to come and go as you please throughout the
event, but here are the times all team members should plan to be on
premises:

\begin{itemize}
\tightlist
\item
  Friday 5-5:30pm - registration
\item
  Friday 5:30pm - kickoff
\item
  Saturday afternoon - group photo. Exact time TBD.
\item
  Sunday 2pm - judging and awards ceremony
\end{itemize}

\hypertarget{food-caffeine}{%
\subsection{Food \& Caffeine}\label{food-caffeine}}

Our goal is to keep you fed \& caffinated all weekend long. Check the
Schedule for mealtimes.

\hypertarget{security-access}{%
\subsection{Security \& Access}\label{security-access}}

Campus security will open the doors promptly at 8am Saturday and Sunday
morning, and come to lock the doors at Midnight Friday \& Saturday. It
is not advised to leave materials in the building although it will be
locked.

\textbf{Teams who are spotted working in areas other than Sylvesters or
after midnight will be disqualified from the competition.}

\hypertarget{social-media}{%
\section{Social Media}\label{social-media}}

Did you know that, this year, DataFest will be held at over 30
universities around the world? DataFest takes place during a six-week
window in the Spring, and different universities hold their event at
different times.

An essential ingredient of Datafest is that the data be a surprise.
There are several reasons for this. One is that it ensures that all
teams start on an equal footing. Another is that, even if you research
the organization that donated the data, you might still be totally
unprepared for the context of the data (and maybe less prepared than had
you known nothing!) Finally, it is simply more fun.

You are encouraged to share your excitement on social media and thank
our sponsors. The official Twitter hashtag is \#datafest19.
However\ldots{}

PLEASE KEEP THE SECRET UNTIL MAY 4, 7pm. Teams that leak the name or any
information about the identify of the data donor will be disqualified!

Please do not post pictures that reveal any
charts/graphs/tables/summaries/models that might reveal the data donor.

\hypertarget{computing-and-supplies}{%
\section{Computing and supplies}\label{computing-and-supplies}}

We recommend that every member of the team bring a laptop, if possible.
You might find it helpful to have a mix of PCs and Macs, since they have
different strengths.

We recommend that you make sure beforehand that the software you will be
using throughout the weekend is properly installed and running on your
computer. You will be working with a large dataset so make sure that you
have the space for it on your hard drive.

You might want to bring some favorite statistical or computational
reference books, if you have them, or bookmark some pages that you
routinely refer to.

We will provide meals, snacks, and munchies. Feel free to bring anything
additional you might want.

\hypertarget{cloud-computing-resources-for-datafest}{%
\subsubsection{Cloud computing resources for
DataFest}\label{cloud-computing-resources-for-datafest}}

RStudio is hosted on the cloud at \url{https://rstudio.cloud/}

You can use a Python instance from Google Colab
\url{https://colab.research.google.com/}

Both of these options are accessible to anyone with a gmail account.

\hypertarget{data}{%
\subsection{Data}\label{data}}

At the end of the kickoff presentation each team should send one member
to the registration desk at Selvester's Cafe and check out a USB stick
containing the data. When you're done downloading the data off it please
return it to the registration desk so that another team can use it.

\hypertarget{large-data-advice}{%
\subsubsection{Large data advice}\label{large-data-advice}}

The dataset you will be working with is quite large. If you type a
variable name to view it, it will take a while to display. Therefore,
remember these R commands: \texttt{head()}, \texttt{tail()},
\texttt{str()}.

We strongly recommend you create a small data set that you can use to
test things on. Then, if it works out, you can apply your procedure to
the large dataset. Some procedures can take a frustratingly long time to
run on large data sets, and so it will be comforting to know that your
procedure works (because you tested it on a smaller data set) while you
wait. We recommend taking a random sample of rows from the original data
set, but there might be other approaches you find useful.

\hypertarget{presentations-judging-and-awards}{%
\section{Presentations, judging, and
awards}\label{presentations-judging-and-awards}}

\hypertarget{presentations}{%
\subsection{Presentations}\label{presentations}}

Each team will have 4 minutes + 1 minute Q\&A to present their findings
to the judges. That's exactly 4 minutes, not 4 minutes and a few
additional seconds. Each team will be allowed at most two slides. Two!
So at some point Saturday night or Sunday morning, you might want to set
aside time to think about what you want the judges to know. The 4 minute
presentation and 1 minute Q\&A time limits will be strictly enforced.
All team members must be present for the presentation, but not all team
members need to actually speak (given the time limitation).

\textbf{Optional}

Along with your presentation you are also allowed to turn in a one-page
write-up of your project. You can think about this as the text of your
presentation. The judges can refer to these during deliberation. You
will upload these documents along with your presentation. Only 1 page
will be printed.

\hypertarget{submitting-your-presentation}{%
\paragraph{Submitting your
presentation}\label{submitting-your-presentation}}

At 1pm on Sunday all work must stop and you must upload your
presentation and your optional write up to the submission website linked
at the top of this page (available starting Sunday AM). If you are
having technical difficulty, you can ask a consultant for help.
Consultants will be around to help.

Teams who fail to upload their presentations and write-ups by 1:0pm on
Sunday will not be eligible to have their presentations judged.

\hypertarget{file-naming}{%
\paragraph{File naming}\label{file-naming}}

The files you're submitting must be named in the following manner:

\begin{itemize}
\tightlist
\item
  {[}Team Name{]} - Presentation
\item
  {[}Team Name{]} - Writeup
\end{itemize}

\hypertarget{allowed-file-formats}{%
\paragraph{Allowed file formats}\label{allowed-file-formats}}

\begin{itemize}
\tightlist
\item
  We very strongly recommend using PDF, Keynote, or PowerPoint.
\item
  If using a web-based tool like GoogleDocs or Prezi, please export to
  PDF and upload the PDF as your submission.
\end{itemize}

Note that you will not have time to log on/off to your account before
your presentation. We don't want to restrict your creativity but it is
your responsibility to make sure that your presentation works seamlessly
before the judging session begins.

\hypertarget{judging}{%
\subsection{Judging}\label{judging}}

The Judges will convene in a side room to deliberate and rank their
nominations. Three of these will be selected for the award categories
listed below. The judges also have the option to name a fourth winner as
Judges' Pick.

All are welcomed to the presentations and award ceremony.

\hypertarget{awards}{%
\subsection{Awards}\label{awards}}

Awards will be given in three categories:

\begin{itemize}
\tightlist
\item
  Best Insight
\item
  Best Use of Outside Data
\item
  Best Visualization
\end{itemize}

These are listed in no particular order.

The judges also have the option to name a fourth winner as Judges' Pick.

Award ceremony will take place in Selvester's Cafe.

Winners will receive medals and books as well as one-year student
memberships to the American Statistical Association. See
\url{http://www.amstat.org/membership/} for membership benefits.

\hypertarget{raffle-prizes}{%
\paragraph{Raffle prizes}\label{raffle-prizes}}

\begin{itemize}
\tightlist
\item
  Throughout the event we will be giving out raffle prizes.
  Announcements for these will be shared on social media or through
  Slack. Follow these channels to get a chance to win one of these sweet
  prizes!
\item
  Winning will also require that you are on premises at the time a prize
  is announced.
\end{itemize}

\hypertarget{recruiting}{%
\subsection{Recruiting}\label{recruiting}}

DataFest is a great recruiting opportunity for many employers, and
surely they won't miss it!

Many of our sponsors are attending the event so you can find out more
about them.

Most of our consultants are coming from companies who are recruiting or
at a minimum wanting to meet you, so chat with them, find out what they
do, network.

We will collect resumes and share them with some of our sponsors.
Participation in the resume book is optional, but highly recommended.
You will receive information about this during the event.

\hypertarget{rules}{%
\section{Rules}\label{rules}}

\begin{itemize}
\item
  You can come and go as you please, but all work must be completed on
  premises.
\item
  Do not use any space other than Sylvesters, or work after midnight.
  Students found working in other areas or after midnight will be
  disqualified from the competition.
\item
  You must follow the Code of Conduct. This can be found at
  \url{https://chicodatafest.netlify.com/faq/} and printed at the event.
\item
  Do not share the name of the data source publicly or on social media
  before May 4th. There are many other upcoming DataFests around the
  country and we want to make sure the dataset remains a surprise for
  them.
\item
  Before your team gets the data all members must have signed
  Non-Disclosure agreement on file at the registration desk. You can
  freely share your results, presentations, findings, etc. as part of
  your digital portfolio, however you are not allowed to share the raw
  data with anyone outside of DataFest. At the end of DataFest, you must
  delete all data from thumb drives, hard drives, etc. The data are
  sensitive.
\item
  As much as possible during the event there will be a friendly
  consultants present. These are faculty, grad students, or other
  professionals from our community. Not all will have field specific
  knowledge on the dataset. They all have different areas of expertise,
  so if you get stuck on something and one consultant isn't able to
  help, ask someone else later. Feel free to ask anything. This is not
  an exam, but a collaboratory competition. Do not expect the
  consultants to write code for you, or do data management, etc. They
  are there to help point you in the right direction, but you're
  responsible for getting there on your own.
\item
  PLEASE KEEP THE SECRET UNTIL MAY 4, 7pm. Teams that leak the name or
  any information about the identify of the data donor will be
  disqualified! Please do not post pictures that reveal any
  charts/graphs/tables/summaries/models that might reveal the data
  donor.
\item
  Please do not tweet, hashtag, Facebook, snapchat, etc about the
  identify of the donor or the context of the data. This means you
  should refrain from any hints, explicit or implicit statements that
  might reveal the context of the data.
\item
  Be careful about github repositories. Make sure yours is invisible to
  the outside world.
\end{itemize}


\end{document}
