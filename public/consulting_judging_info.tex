\documentclass[]{article}
\usepackage{lmodern}
\usepackage{amssymb,amsmath}
\usepackage{ifxetex,ifluatex}
\usepackage{fixltx2e} % provides \textsubscript
\ifnum 0\ifxetex 1\fi\ifluatex 1\fi=0 % if pdftex
  \usepackage[T1]{fontenc}
  \usepackage[utf8]{inputenc}
\else % if luatex or xelatex
  \ifxetex
    \usepackage{mathspec}
  \else
    \usepackage{fontspec}
  \fi
  \defaultfontfeatures{Ligatures=TeX,Scale=MatchLowercase}
\fi
% use upquote if available, for straight quotes in verbatim environments
\IfFileExists{upquote.sty}{\usepackage{upquote}}{}
% use microtype if available
\IfFileExists{microtype.sty}{%
\usepackage{microtype}
\UseMicrotypeSet[protrusion]{basicmath} % disable protrusion for tt fonts
}{}
\usepackage[margin=1in]{geometry}
\usepackage{hyperref}
\hypersetup{unicode=true,
            pdftitle={Consultant \& Judge Information},
            pdfborder={0 0 0},
            breaklinks=true}
\urlstyle{same}  % don't use monospace font for urls
\usepackage{graphicx,grffile}
\makeatletter
\def\maxwidth{\ifdim\Gin@nat@width>\linewidth\linewidth\else\Gin@nat@width\fi}
\def\maxheight{\ifdim\Gin@nat@height>\textheight\textheight\else\Gin@nat@height\fi}
\makeatother
% Scale images if necessary, so that they will not overflow the page
% margins by default, and it is still possible to overwrite the defaults
% using explicit options in \includegraphics[width, height, ...]{}
\setkeys{Gin}{width=\maxwidth,height=\maxheight,keepaspectratio}
\IfFileExists{parskip.sty}{%
\usepackage{parskip}
}{% else
\setlength{\parindent}{0pt}
\setlength{\parskip}{6pt plus 2pt minus 1pt}
}
\setlength{\emergencystretch}{3em}  % prevent overfull lines
\providecommand{\tightlist}{%
  \setlength{\itemsep}{0pt}\setlength{\parskip}{0pt}}
\setcounter{secnumdepth}{0}
% Redefines (sub)paragraphs to behave more like sections
\ifx\paragraph\undefined\else
\let\oldparagraph\paragraph
\renewcommand{\paragraph}[1]{\oldparagraph{#1}\mbox{}}
\fi
\ifx\subparagraph\undefined\else
\let\oldsubparagraph\subparagraph
\renewcommand{\subparagraph}[1]{\oldsubparagraph{#1}\mbox{}}
\fi

%%% Use protect on footnotes to avoid problems with footnotes in titles
\let\rmarkdownfootnote\footnote%
\def\footnote{\protect\rmarkdownfootnote}

%%% Change title format to be more compact
\usepackage{titling}

% Create subtitle command for use in maketitle
\newcommand{\subtitle}[1]{
  \posttitle{
    \begin{center}\large#1\end{center}
    }
}

\setlength{\droptitle}{-2em}

  \title{Consultant \& Judge Information}
    \pretitle{\vspace{\droptitle}\centering\huge}
  \posttitle{\par}
  \subtitle{\url{https://chicodatafest.netlify.com/}}
  \author{}
    \preauthor{}\postauthor{}
    \date{}
    \predate{}\postdate{}
  

\begin{document}
\maketitle

\textbf{Note: This document may be updated as the event approaches; any
major updates will be clearly marked.}

\hypertarget{general-event-info}{%
\subsection{General Event Info}\label{general-event-info}}

All information regarding the event can be found at
\url{https://chicodatafest.netlify.com/}.

\hypertarget{location}{%
\subsection{Location}\label{location}}

Throughout the weekend students will work at CSU, Chico's Selvester's
Cafe. There are many places to park around campus and downtown.
\href{https://www.csuchico.edu/maps/campus/?id=1193\#!m/316229?ce/0,31642,28505?ct/28506,36245,36247,36246,36473,36472,32703,32116,32115,32092,32068,32015,31198?mc/39.730033800326076,-121.84518098831178?z/17?lvl/0}{Here
is a link to an interactive map showing the cafe, and parking nearby.}

\hypertarget{when-you-get-there}{%
\subsubsection{When you get there}\label{when-you-get-there}}

Please check in at registration desk at Selvester's Cafe, and grab your
badge and consultant T-shirt (if you registered early) when you get in.
It's helpful to students for instance, to be able to clearly identify
consultants from participants. Judges will get a notepad and judging
rubric along with their badge.

\hypertarget{contact}{%
\subsection{Contact}\label{contact}}

You are also welcome to join the event Slack workspace. The invitation
link will be emailed along with your schedule If you need to contact an
organizer at any point during the event, email Robin at
\href{mailto:rdonatello@csuchico.edu}{\nolinkurl{rdonatello@csuchico.edu}}
or PM her through Slack.

\hypertarget{data}{%
\subsection{Data}\label{data}}

{Please note that this information is currently TOP SECRET!!!} We do not
want it getting out to anybody until all other locations are done with
their DataFest in later weeks.

All \textbf{judges} will receive a link to a Box document containing
information on the data. Please let Robin know if you have not received
the invite. Any consultants who would like access to this document,
please contact Robin to request access.

The data set itself will \textbf{not} be provided to any consultants.
{We request that all computers be left at home.}

\hypertarget{consultants}{%
\subsection{Consultants}\label{consultants}}

The most important thing is to keep the mood light and encouraging! We
suspect by sometime Saturday afternoon things might seem rather dire to
some of the students.

On Friday night, students will be busy trying to make sense of the data.
We expect that some of them might have technical problems with getting
started (loading the data, viewing it, etc.). Throughout the weekend the
teams will be on their own, though we might have intermittent ``check
ins''. We imagine they will get stuck and need advice. Sometimes, the
advice could be highly technical and, depending on your background,
outside your expertise. Don't worry. They know that you are not there to
solve their problems, but to offer advice. See if you can steer them
towards standard problem-solving techniques: break the problem into
smaller pieces, go online for advice, etc. Guide them to think about
context. What sort of distribution do they expect? Why? What might cause
that? How does that compare to what they saw?

This is a competition, but it is supposed to be friendly and
collaborative, so don't worry about revealing any special knowledge.
This is not an exam, and so if someone asks, please answer if you can,
and don't worry about other teams over hearing. I'm hoping that, after
the first evening, teams will be sharing basic technical advice on their
own.

Everyone is welcome to the presentations and award ceremony.

\hypertarget{a-note-about-software}{%
\subsubsection{A note about software}\label{a-note-about-software}}

Most students will use R or Python but we also expect many to use SAP,
Matlab, SAS, Excel, JMP, etc. as well. If asked for support on a
platform you're not familiar with, simply stating so is sufficient. If
asked for support on a platform you are familiar with, you are welcomed
to assist at any capacity you feel comfortable/up to -- from talking
through the approach at a high level to sitting down and coding with
them.

\hypertarget{judging}{%
\subsection{Judging}\label{judging}}

There are three prize categories:

\begin{itemize}
\tightlist
\item
  Best Insight
\item
  Best Visualization
\item
  Best Use of Outside Data
\end{itemize}

The latter category is meant to encourage students to find supporting
information and/or data beyond what we give them.

An optional ``Judges Choice'' will be available to the judges to use if
desired.

Teams will have 5 minutes (4 min presentation + 1 min Q\&A) and 2 slides
to make their case. Some teams may also submit a 1 page writeup with
their presentation. This paper will be printed out and provided to you
before judging starts.

At least one helper will also be in the room with you to run the
session, keep time, etc. You will also be given a judging pack including
a rubric and notepad to help keep things straight when you pick up your
badge. It's strongly recommended that you make note of teams that are
especially strong as you're watching the presentations so that the
deliberation conversation can be as efficient as possible.

\hypertarget{feedback}{%
\subsection{Feedback}\label{feedback}}

ASA DataFest has become an annual event that is now being officially
sponsored by the American Statistical Association and held at numerous
locations with participation from a large number of universities. We
hope to grow the event further in the coming years. With that in mind,
please send us any advice or constructive criticisms that will help us
improve this event in the future to
\href{mailto:datascience@csuchico.edu}{\nolinkurl{datascience@csuchico.edu}}.

Our primary goal is to provide a rewarding experience for the
undergraduates, an experience that sharpens their analytical skills and
gives them some confidence that they can take what they learn here out
into the real world.


\end{document}
